\documentclass{report}
% \begin{}

% LaTeX Packages %
%%%%%%%%%%%%%%%%%%
\usepackage{etoolbox}

% Text Packages %
%%%%%%%%%%%%%%%%%
\usepackage[spanish]{babel}
\usepackage[inline]{enumitem}
\usepackage{xcolor}

% Math Packages %
%%%%%%%%%%%%%%%%%
\usepackage{mathtools}
\usepackage{amssymb, amsthm}
\usepackage{physics}
\usepackage{bbm}

% Math symbols %
%%%%%%%%%%%%%%%%
\newcommand{\placeholderParameter}{-}
\newcommand{\characteristic}{\mathbbm{1}}
\newcommand{\naturalNumbers}{\mathbb{N}}
\newcommand{\integerNumbers}{\mathbb{Z}}
\newcommand{\realNumbers}{\mathbb{R}}
\newcommand{\complexNumbers}{\mathbb{C}}
\newcommand{\integrableFunctions}{L^1(\realNumbers^n)}
\newcommand{\locallyIntegrableFunctions}{L^1_{\textnormal{loc}}(\realNumbers^n)}
\newcommand{\local}{\textnormal{loc}}
\newcommand{\closure}[1]{\bar{1}}

\newtheorem{theorem}{Teorema}
\newtheorem{definition}{Definición}
\newtheorem{myDefinition}{Definición propia}
\newtheorem{exercise}{Ejercicio}
\newtheorem{myLemma}{Lema propio}
\newtheorem{lemma}{Lema}
\theoremstyle{remark}
\newtheorem{remark}{Observación}

\DeclareMathOperator{\lebesgueMeasure}{\lambda}
\DeclareMathOperator{\support}{\textnormal{sop}}

% Math intervals %
%%%%%%%%%%%%%%%%%%
\newcommand{\leftOpenInterval}{\left]}
\newcommand{\rightOpenInterval}{\right[}
\newcommand{\leftClosedInterval}{\left[}
\newcommand{\rightClosedInterval}{\right]}
\newcommand{\interval}[3]{%
  \ifstrequal{#1}{oo}{%
    \leftOpenInterval #2, #3 \rightOpenInterval%
  }{%
    \ifstrequal{#1}{co}{%
      \leftClosedInterval #2, #3 \rightOpenInterval%
    }{%
      \ifstrequal{#1}{oc}{
        \leftOpenInterval #2, #3 \rightClosedInterval%
      }{%
        \ifstrequal{#1}{cc}{
          \leftClosedInterval #2, #3 \rightClosedInterval
        }{%
        }%
      }%
    }%
  }%
}
% End math intervals %
%%%%%%%%%%%%%%%%%%%%%%

\title{Ecuaciones diferenciales \\ Clase 0}
\author{Pablo Brianese}
\begin{document}
  \maketitle

  \section{Teorema de la Divergencia de Gauss}

  Sea \(n = 2\) y consideremos una función \(\vec{F} : \realNumbers^2 \rightarrow \realNumbers^2\) de clase \(C^1\).
  Podemos escribir en coordenadas \(\vec{F}(x, y) = (f_1(x, y), f_2(x, y))\).
  
  Sea \(R\) una región que puede expresarse como \(\{(x, y) \in \realNumbers^2 :  a < x < b, \phi_1(x) < y < \phi_2(x)\}\), una región delimitada por las gráficas de dos funciones continuamente derivables \(\phi_1, \phi_2 : \interval{cc}{a}{b} \rightarrow \realNumbers\), que satisfacen \(\phi_1(a) = \phi_2(a)\), \(\phi_1(x) < \phi_2(x)\) para todo \(x \in \interval{oo}{a}{b}\), \(\phi_1(b) = \phi_2(b)\).
  Calculemos ahora una primera integral
  \begin{align}
    \int_R \frac{\partial f_2}{\partial y} dy dx
    =
    \int_a^b\int_{\phi_1(x)}^{\phi_2(x)} \frac{\partial f_2}{\partial y} dy dx
    =
    \int_a^b f_2(x, \phi_2(x)) - f_2(x, \phi_1(x)) dx
  \end{align}
  Sea \(\vec{n} = \tilde{n} / \norm{\tilde{n}}\) la normal exterior de la frontera \(\partial R\),
  donde
  \begin{align}
    \tilde{n}
    =
    \left\{
      \begin{aligned}
        (- \phi_2'(x), 1)
        && \text{en }
        \Gamma_2 = \{(x, \phi_2(x)) : x \in \interval{cc}{a}{b}\}
        \\
        (\phi_1'(x), - 1)
        && \text{en }
        \Gamma_1 = \{(x, \phi_1(x)) : x \in \interval{cc}{a}{b}\}
      \end{aligned}
    \right.
  \end{align}
  con
  \begin{align}
    \norm{\tilde{n}}
    =
    \left\{
      \begin{aligned}
        \sqrt{1 + \phi_2'(x)^2}
        && \text{en }
        \Gamma_2 = \{(x, \phi_2(x)) : x \in \interval{cc}{a}{b}\}
        \\
        \sqrt{1 + \phi_1'(x)^2}
        && \text{en }
        \Gamma_1 = \{(x, \phi_1(x)) : x \in \interval{cc}{a}{b}\}
      \end{aligned}
    \right.
  \end{align}
  Entonces, si \(\vec{n} = (n_1, n_2)\), se tiene 
  \begin{align}
    n_2
    =
    \left\{
      \begin{aligned}
        (1 + \phi_2'(x)^2)^{- 1 / 2}
        && \text{en }
        \Gamma_2 = \{(x, \phi_2(x)) : x \in \interval{cc}{a}{b}\}
        \\
        - (1 + \phi_1'(x)^2)^{- 1 / 2}
        && \text{en }
        \Gamma_1 = \{(x, \phi_1(x)) : x \in \interval{cc}{a}{b}\}
      \end{aligned}
    \right.
  \end{align}
  Esto nos permite calcular una segunda integral
  \begin{align}
    \int_{\partial R} f_2 n_2 d \sigma(x)
    &=
    \int_{\Gamma_1} f_2 n_2 d \sigma(x)
    + \int_{\Gamma_2} f_2 n_2 d \sigma(x)
    \\
    &=
    \int_a^b
      - \frac{f_2(x, \phi_1(x))}{\sqrt{1 + \phi_1'(x)^2}}
    \left( \sqrt{1 + \phi_1'(x)^2} d x \right)
    \\
    &\quad
      + \int_a^b
        \frac{f_2(x, \phi_2(x))}{\sqrt{1 + \phi_2'(x)^2}}
      \left( \sqrt{1 + \phi_2'(x)^2} d x \right)
    \\
    &=
    \int_a^b f_2(x, \phi_2(x)) - f_2(x, \phi_1(x)) dx
  \end{align}
  Comparando la primera y la segunda integral, resulta
  \begin{align}
    \label{equation:yAxis2dDivergenceTheorem}
    \int_R \frac{\partial f_2}{\partial y} dy dx
    =
    \int_{\partial R} f_2 n_2 d \sigma
  \end{align}

  Supongamos que \(R\) también se puede expresar como \(\{(x, y) \in \realNumbers^2 : y \in \interval{oo}{c}{d}, \psi_1(y) < x < \psi_2(y)\}\), donde las funciones \(\psi_1, \psi_2 : \interval{cc}{c}{d} \rightarrow \realNumbers\) son continuamente derivables y satisfacen \(\psi_1(c) = \psi_2(c)\), \(\psi_1(y) < \psi_2(y)\) para todo \(y \in \interval{oo}{c}{d}\), \(\psi_1(b) = \psi_2(b)\).
  Cálculos análogos a los anteriores prueban que
  \begin{align}
    \label{equation:xAxis2dDivergenceTheorem}
    \int_R \frac{\partial f_1}{\partial x} dx dy
    =
    \int_{\partial R} f_1 n_1 d \sigma
  \end{align}

  Por tanto, de \eqref{equation:yAxis2dDivergenceTheorem} y de \eqref{equation:xAxis2dDivergenceTheorem} se tiene
  \begin{align}
    \label{equation:GaussDivergenceTheorem}
    \int_R (\nabla \cdot \vec{F}) dx dy
    =
    \int_{\partial R} (\vec{F} \cdot \vec{n}) d \sigma
  \end{align}
  La idea intuitiva es que la \emph{variación} de \(\vec{F}\) en \(R\) es igual a la \emph{cantidad} de \(\vec{F}\) que entra o sale a través de \(\partial R\).
  El segundo término de la ecuación, se llama \emph{flujo del campo \(\vec{F}\) a través de \(\partial R\)}.

  Tanto el flujo como la divergencia tienen perfecto sentido en dimensiones mayores;
  nos proponemos dar condiciones suficientes sobre los dominios \(R \subseteq \realNumbers^n\), de forma que la expresión \eqref{equation:GaussDivergenceTheorem} sea válida.
  El caso más sencillo para el que se tiene la igualdad \eqref{equation:GaussDivergenceTheorem} es el siguiente.

  Consideremos \(\vec{F} = (f_1, \dots, f_n)\) campo de vectores en \(\realNumbers^n\) y supongamos que
  \begin{enumerate*}
    \item \(f_i\) tiene derivadas primeras continuas para \(i = 1, \dots, n\);
    \item existe una bola \(B\) cerrada tal que \(f_i(x) = 0\) si \(x \notin B\), para \(i = 1, \dots, n\).
  \end{enumerate*}
  Bajo estas condiciones podemos probar que 
  \begin{align}
    \label{equation:compactlySupportedFieldsHaveNullTotalDivergence}
    \int_{\realNumbers^n} \nabla \cdot \vec{F} dx = 0
  \end{align}
  En efecto, por definición
  \(
    \nabla \cdot \vec{F}
    =
    \sum_{i = 1}^n
    \frac{\partial f_i}{\partial x_i}
  \).
  Consideremos un \(i \in \{1, \dots, n\}\) arbitrario.
  Usando el teorema de Fubini para funciones integrables (de soporte compacto, en nuestro caso) deducimos que
  \begin{align}
    \int_{\realNumbers^n} \frac{\partial f_i}{\partial x_i} dx
    &=
    \sum_{i = 1}^n
    \int_{\realNumbers} \cdots \int_{\realNumbers}
      \frac{\partial f_i}{\partial x_i}(x_1, \dots, x_n)
    dx_1 \cdots dx_n
    \\
    \label{equation:integralsCanBePermutedBecauseOfCompactSupport}
    &=
    \int_{\realNumbers} \cdots \int_{\realNumbers} \int_{\realNumbers}
      \frac{\partial f_i}{\partial x_i}(x_1, \dots, x_i, \dots, x_n)
    dx_{i} dx_{\tau(2)} \cdots dx_{\tau(n)}
  \end{align}
  donde \(\tau\) es una permutación del conjunto \(\{1, \dots, n\}\) con \(\tau(1) = i\).
  Pero, dado que \(f_i\) se anula fuera de la bola \(B\), el teorema fundamental del cálculo implica
  \begin{align}
    \label{equation:integralIsNullBecauseOfCompactSupport}
    \int_{\realNumbers}
      \frac{\partial f_i}{\partial x_i}(x_1, \dots, x_i, \dots, x_n)
    dx_{i}
    =
    0
  \end{align}
  Luego \(\int_{\realNumbers^n} \frac{\partial f_i}{\partial x_i} dx = 0\).
  Siendo \(i\) arbitrario, se sigue que es nula la integral sobre \(\realNumbers^n\) de la divergencia del campo \(\vec{F}\).
  


  \newpage
  La segunda observación elemental que vamos a hacer posteriormente para hacer una demostración con cierta generalidad, es la siguiente.
  Consideremos \(U \subseteq \realNumbers^{n - 1}\) un abierto acotado y \(\phi \in C^1(U)\) una función positiva.
  Llamamos \(S\) a la gráfica de \(\phi\) pensada como subconjunto de \(\realNumbers^n\);
  es decir,
  \begin{align}
    S
    &=
    \{(\overline{x}, \phi(\overline{x})) : \overline{x} \in U\}
    \\
    &=
    \{(x_1, \dots, x_{n - 1}, \phi(x_1, \dots, x_{n - 1})) : (x_1, \dots, x_{n - 1}) \in U\}
  \end{align}
  Dada esta descripción paramétrica, podemos describir el espacio tangente a \(S\) en un punto \(p \in S\) como aquel generado por el conjunto de vectores tangentes
  \begin{align}
    \left\{
      \left.
        \frac{\partial}{\partial x_i}
        (\overline{x}, \phi(\overline{x}))
      \right\vert_{\overline{x} = p}
    \right\}_{i = 1}^{n - 1}
    &=
    \left\{
      e_i
      +
      \frac{\partial \phi}{\partial x_i}(\overline{x}) e_n
    \right\}_{i = 1}^{n - 1}
  \end{align}
  Encontramos que el vector \((- \phi_{x_1}(\overline{x}), \dots, - \phi_{x_{n - 1}}(\overline{x}), 1)\) es ortogonal a cada uno de ellos.
  Eso nos permite calcular \(\nu\), normal a \(S\) con la componente \(\nu_n(\overline{x}) > 0\), resultando
  \begin{align}
    \nu(p)
    =
    \frac{1}{(1 + \norm{\nabla \phi (\overline{x})}^2)^{1/2}}
    \left(
      - \frac{\partial \phi}{\partial x_1}(\overline{x}), \dots, - \frac{\partial \phi}{\partial x_{n - 1}}(\overline{x}),
      1
    \right)
  \end{align}
  
  Consideremos ahora el conjunto \emph{cilíndrico} \(C\), definido por
  \begin{align}
    C
    &=
    \{(\overline{x}, x_n) : \overline{x} \in U, x_n \in \interval{cc}{0}{\phi(\overline{x})}\}
  \end{align}
  Si se toma \(f\) con derivadas primeras continuas en \(C\), tal que \(f(\overline{x}, 0) = 0\), se tiene
  \begin{align}
    \label{equation:divergenceTheoremInACylinder}
    \int_C \frac{\partial f}{\partial x_n}(x) dx
    =
    \int_S f \nu_n d \sigma
  \end{align}
  En efecto, simplificar el lado izquierdo de la ecuación es sencillo
  \begin{align}
    \int_C \frac{\partial f}{\partial x_n}(x) dx
    =
    \int_U
      \int_0^{\phi(\overline{x})}
        \frac{\partial f}{\partial x_n} (\overline{x}, x_n)
      d x_n
    d \overline{x}
    =
    \int_U
      f(\overline{x}, \phi(\overline{x}))
    d \overline{x}
  \end{align}
  Mientras que la misma tarea es compleja para la integral de superficie del lado derecho.
  Por definición
  \begin{align}
    \int_S f \nu_n d \sigma
    &=
    \int_U
      f(\overline{x}, \phi(\overline{x}))
      \nu_n(\overline{x}, \phi(\overline{x}))
      g(\overline{x})
    d \overline{x}
  \end{align}
  donde la misma definición dice
  \begin{align}
    g(\overline{x})
    =
    \text{abs det}
    \begin{pmatrix}
      \nu(\overline{x}, \phi(\overline{x}))
      \\
      e_1
      +
      \frac{\partial \phi}{\partial x_1}(\overline{x})
      e_n
      \\
      \vdots
      \\
      e_{n - 1}
      +
      \frac{\partial \phi}{\partial x_{n - 1}}(\overline{x})
      e_n
    \end{pmatrix}
  \end{align}
  Por construcción
  \(
    \nu_n (\overline{x}, \phi(\overline{x}))
    =
    \left(1 + \norm{\nabla \phi (\overline{x})}^2\right)^{- 1 / 2}
  \).
  Pero el cálculo de \(g\) es complejo e involucra un proceso inductivo.
  Comenzamos con
  \begin{align}
    &g(\overline{x})
    \\
    &=
    \text{abs} \det
    \begin{pmatrix}
      \nu(\overline{x}, \phi(\overline{x}))
      \\
      e_1
      +
      \frac{\partial \phi}{\partial x_1}(\overline{x})
      e_n
      \\
      \vdots
      \\
      e_{n - 1}
      +
      \frac{\partial \phi}{\partial x_{n - 1}}(\overline{x})
      e_n
    \end{pmatrix}
    \\
    &=
    \text{abs} \det
    \begin{pmatrix}
      \left( 1 + \norm{\nabla \phi}^2 \right)^{- 1 / 2}
    \left(
      - \frac{\partial \phi}{\partial x_1}, \dots, - \frac{\partial \phi}{\partial x_{n - 1}},
      1
    \right)
      \\
      \left(
        1,
        0, \dots, 0,
        \frac{\partial \phi}{\partial x_1}
      \right)
      \\
      \vdots
      \\
      \left(
        0, \dots, 0,
        1,
        \frac{\partial \phi}{\partial x_{n - 1}}
      \right)
    \end{pmatrix}
    \\
    &=
    \frac{1}{\left( 1 + \norm{\nabla \phi}^2 \right)^{1/2}}
    \text{abs} \det
    \begin{pmatrix}
      - \phi_{x_1}
      &- \phi_{x_2}
      &\dots
      &- \phi_{x_{n - 2}}
      &- \phi_{x_{n - 1}}
      &1
      \\
      1
      &0
      &\dots
      &0
      &0
      &\phi_{x_1}
      \\
      0
      &1
      &\dots
      &0
      &0
      &\phi_{x_2}
      \\
      \vdots
      &\vdots
      &
      &\vdots
      &\vdots
      &\vdots
      \\
      0
      &0
      &\dots
      &1
      &0
      &\phi_{x_{n - 2}}
      \\
      0
      &0
      &\dots
      &0
      &1
      &\phi_{x_{n - 1}}
    \end{pmatrix}
  \end{align}
  Con el objetivo de facilitar la comprensión del cálculo, definimos \(a_i = - \phi_{x_i} (\overline{x})\) para \(i \in \{1, \dots, n - 1\}\).
  Calculamos el determinante a lo largo de su primera columna
  \begingroup
  \allowdisplaybreaks
  \begin{align}
    &\begin{vmatrix}
      - a_1
      &- a_2
      &\dots
      &- a_{n - 2}
      &- a_{n - 1}
      &1
      \\
      1
      &0
      &\dots
      &0
      &0
      &a_1
      \\
      0
      &1
      &\dots
      &0
      &0
      &a_2
      \\
      \vdots
      &\vdots
      &
      &\vdots
      &\vdots
      &\vdots
      \\
      0
      &0
      &\dots
      &1
      &0
      &a_{n - 2}
      \\
      0
      &0
      &\dots
      &0
      &1
      &a_{n - 1}
    \end{vmatrix}
    \\
    &=
    (- a_1)
    \begin{vmatrix}
      \color{red} - a_1
      &\color{red} - a_2
      &\color{red} \dots
      &\color{red} - a_{n - 2}
      &\color{red} - a_{n - 1}
      &\color{red} 1
      \\
      \color{red} 1
      &0
      &\dots
      &0
      &0
      &a_1
      \\
      \color{red} 0
      &1
      &\dots
      &0
      &0
      &a_2
      \\
      \color{red} \vdots
      &\vdots
      &
      &\vdots
      &\vdots
      &\vdots
      \\
      \color{red} 0
      &0
      &\dots
      &1
      &0
      &a_{n - 2}
      \\
      \color{red} 0
      &0
      &\dots
      &0
      &1
      &a_{n - 1}
    \end{vmatrix}
    \\
    &\quad
      - (1)
      \begin{vmatrix}
        \color{red} - a_1
        &- a_2
        &\dots
        &- a_{n - 2}
        &- a_{n - 1}
        &1
        \\
        \color{red} 1
        &\color{red} 0
        &\color{red} \dots
        &\color{red} 0
        &\color{red} 0
        &\color{red} a_1
        \\
        \color{red} 0
        &1
        &\dots
        &0
        &0
        &a_2
        \\
        \color{red} \vdots
        &\vdots
        &
        &\vdots
        &\vdots
        &\vdots
        \\
        \color{red} 0
        &0
        &\dots
        &1
        &0
        &a_{n - 2}
        \\
        \color{red} 0
        &0
        &\dots
        &0
        &1
        &a_{n - 1}
      \end{vmatrix}
      \\
    &=
    - a_1
    \begin{vmatrix}
      0
      &0
      &\dots
      &0
      &0
      &a_1
      \\
      1
      &0
      &\dots
      &0
      &0
      &a_2
      \\
      0
      &1
      &\dots
      &0
      &0
      &a_3
      \\
      \vdots
      &\vdots
      &
      &\vdots
      &\vdots
      &\vdots
      \\
      0
      &0
      &\dots
      &1
      &0
      &a_{n - 2}
      \\
      0
      &0
      &\dots
      &0
      &1
      &a_{n - 1}
    \end{vmatrix}
    -
    \begin{vmatrix}
      - a_2
      &- a_3
      &\dots
      &- a_{n - 2}
      &- a_{n - 1}
      &1
      \\
      1
      &0
      &\dots
      &0
      &0
      &a_2
      \\
      0
      &1
      &\dots
      &0
      &0
      &a_3
      \\
      \vdots
      &\vdots
      &
      &\vdots
      &\vdots
      &\vdots
      \\
      0
      &0
      &\dots
      &1
      &0
      &a_{n - 2}
      \\
      0
      &0
      &\dots
      &0
      &1
      &a_{n - 1}
    \end{vmatrix}
  \end{align}
  \endgroup
  Sumando multiplos apropiados de la primera fila a las demás, el primer término se simplifica notablemente sin alterar el resultado del determinante (por propiedades de las llamadas matrices elementales)
  \begin{align}
    \begin{vmatrix}
      0
      &0
      &\dots
      &0
      &0
      &a_1
      \\
      1
      &0
      &\dots
      &0
      &0
      &a_2
      \\
      0
      &1
      &\dots
      &0
      &0
      &a_3
      \\
      \vdots
      &\vdots
      &
      &\vdots
      &\vdots
      &\vdots
      \\
      0
      &0
      &\dots
      &1
      &0
      &a_{n - 2}
      \\
      0
      &0
      &\dots
      &0
      &1
      &a_{n - 1}
    \end{vmatrix}
    &=
    \begin{vmatrix}
      0
      &0
      &\dots
      &0
      &0
      &a_1
      \\
      1
      &0
      &\dots
      &0
      &0
      &0
      \\
      0
      &1
      &\dots
      &0
      &0
      &0
      \\
      \vdots
      &\vdots
      &
      &\vdots
      &\vdots
      &\vdots
      \\
      0
      &0
      &\dots
      &1
      &0
      &0
      \\
      0
      &0
      &\dots
      &0
      &1
      &0
    \end{vmatrix}
    \\
    &=
    (-1)^{n - 2}
    \begin{vmatrix}
      a_1
      &0
      &0
      &\dots
      &0
      &0
      \\
      0
      &1
      &0
      &\dots
      &0
      &0
      \\
      0
      &0
      &1
      &\dots
      &0
      &0
      \\
      \vdots
      &\vdots
      &\vdots
      &
      &\vdots
      &\vdots
      \\
      0
      &0
      &0
      &\dots
      &1
      &0
      \\
      0
      &0
      &0
      &\dots
      &0
      &1
    \end{vmatrix}
    \\
    &=
    (- 1)^{n - 2} a_1
  \end{align}
  Esa simplificación nos lleva a la identidad
  \begin{align}
    &\begin{vmatrix}
      - a_1
      &- a_2
      &\dots
      &- a_{n - 2}
      &- a_{n - 1}
      &1
      \\
      1
      &0
      &\dots
      &0
      &0
      &a_1
      \\
      0
      &1
      &\dots
      &0
      &0
      &a_2
      \\
      \vdots
      &\vdots
      &
      &\vdots
      &\vdots
      &\vdots
      \\
      0
      &0
      &\dots
      &1
      &0
      &a_{n - 2}
      \\
      0
      &0
      &\dots
      &0
      &1
      &a_{n - 1}
    \end{vmatrix}
    \\
    &=
    (- 1)^{n - 1} a_1^2
    -
    \begin{vmatrix}
      - a_2
      &\dots
      &- a_{n - 2}
      &- a_{n - 1}
      &1
      \\
      1
      &\dots
      &0
      &0
      &a_2
      \\
      \vdots
      &
      &\vdots
      &\vdots
      &\vdots
      \\
      0
      &\dots
      &1
      &0
      &a_{n - 2}
      \\
      0
      &\dots
      &0
      &1
      &a_{n - 1}
    \end{vmatrix}
  \end{align}
  Que nos deja ver una cierta regularidad, un cierto patrón.
  Para describirlo, definimos \(D_1, \dots, D_m\) como
  \begin{align}
    D_k
    &=
    \begin{vmatrix}
      - a_k
      &- a_{k + 1}
      &\dots
      &- a_{n - 2}
      &- a_{n - 1}
      &1
      \\
      1
      &0
      &\dots
      &0
      &0
      &a_k
      \\
      0
      &1
      &\dots
      &0
      &0
      &a_{k + 1}
      \\
      \vdots
      &\vdots
      &
      &\vdots
      &\vdots
      &\vdots
      \\
      0
      &0
      &\dots
      &1
      &0
      &a_{n - 2}
      \\
      0
      &0
      &\dots
      &0
      &1
      &a_{n - 1}
    \end{vmatrix}
  \end{align}
  Repetir el cálculo del determinante a lo largo de la primera columna, arroja
  \begin{align}
    D_k = (- 1)^{n - k} a_k^2 - D_{k + 1},
  \end{align}
  para los primeros términos de la sucesión, con \(k \in \{1, \dots, n - 2\}\).
  Y el último puede calcularse de forma exacta
  \begin{align}
    D_{n - 1} 
    =
    \begin{vmatrix}
      - a_{n - 1}^2 &1 \\
      1 &- a_{n - 1}
    \end{vmatrix}
    =
    - a_{n - 1}^2 - 1
    =
    (- 1)^{n - (n - 1)} (a_{n - 1}^2 + 1)
  \end{align}
  Luego
  \begin{align}
    D_{n - 1}
    &=
    (- 1)^{n - (n - 1)} (a_{n - 1}^2 + 1)
    \\
    D_{n - 2}
    &=
    (- 1)^{n - (n - 2)} a_{n - 2}^2 - (- 1)^{n - (n - 1)} (a_{n - 1}^2 + 1)
    \\
    &=
    (- 1)^{n - (n - 2)} (a_{n - 2}^2 + a_{n - 1}^2 + 1)
    \\
    \vdots\nonumber
    \\
    D_k
    &=
    (- 1)^{n - k} a_k^2 - (- 1)^{n - (k + 1)} (a_{k + 1}^2 + \cdots + a_{n - 1}^2 + 1)
    \\
    &=
    (- 1)^{n - k} (a_{k}^2 + \cdots + a_{n - 1}^2 + 1)
    \\
    \vdots\nonumber
    \\
    D_1
    &=
    (-1)^{n - 1} (a_1^2 + \cdots + a_{n - 1}^2 + 1)
  \end{align}
  Es decir
  \begin{align}
    &\det
    \begin{pmatrix}
      - \frac{\partial \phi}{\partial x_1}
      &- \frac{\partial \phi}{\partial x_2}
      &\dots
      &- \frac{\partial \phi}{\partial x_{n - 2}}
      &- \frac{\partial \phi}{\partial x_{n - 1}}
      &1
      \\
      1
      &0
      &\dots
      &0
      &0
      &\frac{\partial \phi}{\partial x_1}
      \\
      0
      &1
      &\dots
      &0
      &0
      &\frac{\partial \phi}{\partial x_2}
      \\
      \vdots
      &\vdots
      &
      &\vdots
      &\vdots
      &\vdots
      \\
      0
      &0
      &\dots
      &1
      &0
      &\frac{\partial \phi}{\partial x_{n - 2}}
      \\
      0
      &0
      &\dots
      &0
      &1
      &\frac{\partial \phi}{\partial x_{n - 1}}
    \end{pmatrix}
    \\
    &=
    (-1)^{n - 1} \left( \phi_{x_1}^2 + \cdots + \phi_{x_{n -1}}^2 + 1 \right)
    \\
    &=
    (-1)^{n - 1} \left( \norm{\nabla \phi}^2 + 1 \right)
  \end{align}
  Por lo tanto
  \begin{align}
    g(\overline{x})
    =
    \left( \norm{\nabla \phi}^2 + 1 \right)^{1 / 2}
  \end{align}
  Aún más importante
  \begin{align}
    \nu_n(\overline{x}, \phi(\overline{x}))
    g(\overline{x})
    =
    1
  \end{align}
  Luego
  \begin{align}
    \int_S f \nu_n d \sigma
    &=
    \int_U
      f(\overline{x}, \phi(\overline{x}))
    d \overline{x}
  \end{align}

  \newpage
  Las dos observaciones anteriores van a permitir demostrar un \emph{teorema de la divergencia de tipo local} en dominios con algunas condiciones que se precisarán rápidamente y que llamaremos `regulares'.
  Se probará:
  Dado un dominio regular, existen entornos de sus puntos, tales que la el teorema de la divergencia de Gauss se verifica para campos continuamente diferenciales soportados en dichos entornos.

  \emph{Soportado en un entorno} significa que se hacen cero fuera de tal entorno.
  Se define el \emph{soporte} de una función \(f\) como el cierre de los puntos \(x\) tales que \(f(x) \neq 0\), es decir,
  \begin{align}
    \support(f)
    =
    \overline{\{x : f(x) \neq 0\}}
  \end{align}

  Hay que explicar ahora lo que entenderemos por un dominio regular y lo que se entiende por entorno de los puntos del dominio.

  Dado \(D \subseteq \realNumbers^n\), diremos que es un \emph{dominio} si es abierto y es conexo;
  \(\partial D\) denotará su frontera.

  \begin{definition}
    Un dominio acotado \(D \subseteq \realNumbers^n\) se dice que es regular si para cada \(x_0 \in \partial D\) existe un entorno \(U\) de \(x_0\) en \(\realNumbers^n\) y una función \(\phi : U \rightarrow \realNumbers\) continuamente diferenciable, de forma que
    \begin{enumerate}
      \item \(\nabla \phi(x) \neq 0\) si \(x \in U\);
      \item \(\partial D \cap U = \{x \in U : \phi(x) = 0\}\);
      \item \(D \cap U = \{x \in U : \phi(x) < 0\}\)
    \end{enumerate}
  \end{definition}

  Lo que establece la definición anterior es que un dominio regular tiene su frontera definida localmente por ceros de funciones continuamente diferenciales, es decir, por trozos de superficies diferenciables en \(\realNumbers^n\) definidas implicitamente.

  Sea \(D\) un dominio regular y sea \(x \in \partial D\).
  Un vector normal a \(\partial D\) en \(x\) viene dado por \(n = \nabla \phi(x)\), donde \(\phi\) es una función que define \(\partial D\) en el entorno de \(x\) como en la definición de regularidad.
  Diremos que \(n\) es \emph{normal exterior} a \(\partial D\) en \(x\) si para \(\delta > 0\) suficientemente pequeño y \(0 < t < \delta\) se verifican
  \begin{align}
    &x - t n \in D
    &&x + t n \in \realNumbers^n \setminus D
  \end{align}
  Para terminar, denotaremos por \(\nu(x)\) a la normal exterior unitaria en \(x\) a \(\partial D\).
  Si \(D\) es regular, entonces \(\nu(x)\) es un campo continuo en \(\partial D\).
  Compruebe el lector este extremo.
  % ????????????????


  \begin{lemma}
    Sea \(D\) un dominio regular en \(\realNumbers^n\) y \(\vec{F}\)  un campo continuamente diferenciable.
    Si \(T\) es una traslación (\(T(x) = x + t\) para todo \(x \in \realNumbers^n\) donde \(t \in \realNumbers^n\)), entonces
    \begin{align}
      \int_D
        \nabla \cdot \vec{F}
      dx
      =
      \int_{T D}
        \nabla \cdot (F_i \circ T^{- 1})
      dx
    \end{align}
  \end{lemma}
  \begin{proof}
    Por el teorema de cambio de variables, para todo \(i \in \{1, \dots, n\}\)
    \begin{align}
      \int_D \frac{\partial F_i}{\partial x_i}(x) dx
      &=
      \int_{T D}
        \frac{\partial F_i}{\partial x_i}(T^{- 1} u)
        \det\left(D_u T^{-1}\right)
      du
      \\
      &=
      \int_{T D}
        \frac{\partial F_i}{\partial x_i}(u - t)
        \det(I)
      du
      \\
      &=
      \int_{T D}
        \frac{\partial F_i}{\partial x_i}(u - t)
      du
    \end{align}
  \end{proof}

  \begin{lemma}
    Sea \(D\) un dominio regular en \(\realNumbers^n\) y \(\vec{F}\) un campo continuamente diferenciable.
    Si \(T\) es una traslación (\(T(x) = x + t\) para todo \(x \in \realNumbers^n\) donde \(t \in \realNumbers^n\)), entonces
    \begin{align}
      \int_{\partial D} \vec{F} \cdot \vec{\nu} d \sigma
      =
      \int_{\partial (T D)} (\vec{F} \circ T^{- 1}) \cdot \nu d \sigma
    \end{align}
  \end{lemma}
  \begin{proof}
    Tenemos \(T(\partial D) = \partial (T D)\) porque \(T : \realNumbers^n \rightarrow \realNumbers^n\) es un difeomorfismo.
    Luego, si \(\phi : U \subseteq \realNumbers^n \rightarrow \realNumbers^n\)

  
  
  \end{proof}

  \begin{lemma}
    \label{lemma:localDivergenceTheorem}
    Sea \(D\) un dominio regular en \(\realNumbers^n\) con clausura \(\overline{D} = D \cup \partial D\) compacta.
    Entonces dado \(x_0 \in \overline{D}\) existe un entorno \(U\) de \(x_0\) tal que para todo campo \(\vec{F}\) continuamente diferenciable y soportado en \(U\), se verifica
    \begin{align}
      \int_D \nabla \cdot \vec{F} d x
      =
      \int_{\partial D} \vec{F} \cdot \nu d \sigma
    \end{align}
  \end{lemma}
  \begin{proof}
    En un primer momento supondremos que \(x_0 \in D\), y probaremos el lema solo en este caso.
    Si \(x_0 \in D\), entonces podemos encontrar un \(r > 0\) tal que 
    \begin{align}
      B(x_0, r)
      =
      \{
        x \in \realNumbers^n : \abs*{x - x_0} < r
      \}
      \subseteq
      D
    \end{align}
    Sea \(\vec{F}\) un campo continuamente diferenciable en \(D\).
    Si su soporte se encuentra contenido en \(B(x_0, r)\), entonces \(\support \vec{F}\) se encuentra a una distacia positiva del complemento de dicha bola.
    Luego, extender \(\vec{F} : D \rightarrow \realNumbers^n\) por cero fuera de la bola, lo convierte en un campo \(\tilde{F} : \realNumbers^n \rightarrow \realNumbers^n\) continuamente diferenciable y con soporte compacto.
    Por \eqref{equation:compactlySupportedFieldsHaveNullTotalDivergence} para tal campo se tiene
    \begin{align}
      \int_{\realNumbers^n} \nabla \cdot \tilde{F} dx
      =
      \int_D \nabla \cdot \vec{F} dx
      =
      0
      =
      \int_{\partial D} \vec{F} \cdot \nu d \sigma
    \end{align}
    Es decir, si \(x_0 \in D\) se tiene demostrado el lema.

    Supongamos ahora que \(x_0 \in \partial D\).
    La estrategia es probar que fijada una dirección coordenada, \(i = 1, \dots, n\) se puede construir un entorno \(U_i\) en la topología relativa a la de \(\realNumbers^n\) en \(\overline{D}\), tal que si un campo \(\vec{F}\) continuamente diferenciable en \(D\), está soportado en \(U_i\), entonces
    \begin{align}
      \label{equation:divergenceTheoremInTheIthDirection}
      \int_{\partial D}
        F_i \nu_i
      d \sigma
      =
      \int_D
        \frac{\partial F_i}{\partial x_i}
      dx
    \end{align}
    siendo \(F_i\) la coordenada \(i\)--ésima de \(\vec{F}\) y \(\nu_i\) la correspondiente componente de la normal exterior a \(\partial D\).
    De esta forma considerando \(U = \bigcap_{i = 1}^n U_i\), se tendrá probado el lema.

    Probaremos para \(i = n\), tratándose de igual manera el resto de las coordenadas.
    Una observación adicional es necesaria.
    Si sometemos a giros y traslaciones al dominio \(D\) los dos miembros 
    \(
      \int_{\partial D}
        F_i \nu_i
      d \sigma
    \),
    \(
      \int_D
        \frac{\partial F_i}{\partial x_i}
      dx
    \)
    permanecen invariantes.  % ????????????????
    Por tanto, podemos suponer que la coordenada \(n\)--ésima de \(x_0\) es positiva y que \(\nu_n(x_0) > 0\).

    Como, por hipótesis, \(D\) es regular existe \(W\) entorno de \(x_0\) y \(\phi : W \rightarrow \realNumbers\) continuamente diferenciable tal que
    \begin{align}
      \nabla \phi (x_0) \neq 0,
      &&\partial D \cap W
      =
      \{x \in W : \phi(x) = 0\},
      &&D \cap W
      =
      \{x \in W : \phi(x) < 0\}
    \end{align}
    Por el teorema de la función implícita existen, un abierto \(V \subseteq \realNumbers^{n - 1}\) entorno de \(\overline{x_0} = (x_1, \dots, x_{n - 1})\),
    y una \(\psi : V \rightarrow \realNumbers\) continuamente diferenciable, tales que \(\phi(\overline{x}, \psi(\overline{x})) = 0\) para \(\overline{x} \in V\).
    Haciendo \(V\) pequeño si es preciso se tiene por continuidad que \(\psi(\overline{x}) > 0\) \((\forall \overline{x} \in V)\) dado que, por hipótesis \((x_0)_n = \psi(\overline{x}) > 0\).
    Para \(V\) elegido verificando lo anterior consideramos la parte de \(\partial D\) dada por el grafo de \(\psi\), es decir,
    \(S = \{
      (\overline{x}, \psi(\overline{x})) : \overline{x} \in V
    )\}\).

    % no entiendo la relación entre \nu y \partial \phi
    Además, como \(\nu_n(x_0) > 0\), se tiene, en particular \(\frac{\partial \phi}{\partial x_n}(x_0) > 0\).
    Y por tanto existe \(\delta > 0\) tal que, si \(\abs*{\overline{x} - \overline{x}_0} < \delta\) y \(\abs*{x_n - (x_0)_n} < \delta\),
    entonces \(\frac{\partial \phi}{\partial x_n}(\overline{x}, x_n) > 0\).
    Se toma \(\delta\) suficientemente pequeño para que \(B = \{\overline{x} : \abs*{\overline{x} - \overline{x}_0} < \delta\} \subseteq V\).
    Llamemos \(R = \{
      (\overline{x}, x_n) :
      \abs{\overline{x} - \overline{x}_0} < \delta,
      \abs{x_n - (x_0)_n} < \delta
    \}\)
    y definamos el intervalo \(I_{\overline{x}} = \{x_n : (\overline{x}, x_n) \in R\}\), entonces, en particular, la función \(g_{\overline{x}}(x_n) = \phi(\overline{x}, x_n)\) es creciente en \(I_{\overline{x}}\).
    Además, para cada \(\overline{x}\) el intervalo \(I_{\overline{x}}\) 
    contiene el valor \(\psi(\overline{x})\), que es donde \(\phi(\overline{x}, \psi(\overline{x})) = 0\).
    Como \(\phi(\overline{x}, \cdot)\) es creciente como función de \(x_n\) en \(I_{\overline{x}}\), resulta que
    \begin{align}
      D \cap R
      =
      \{(\overline{x}, x_n) \in R : x_n < \psi(\overline{x})\}
    \end{align}
    Dicho de otra manera, \(D \cap R\) está contenido en un conjunto cilíndrico como el utilizado en la ecuación \eqref{equation:divergenceTheoremInACylinder} y así se prueba \eqref{equation:divergenceTheoremInTheIthDirection} para los campos continuamente diferenciables en \(D\) soportados en \(D \cap R\).
  \end{proof}


  \newpage
  Queda ahora por globalizar el resultado y así obtener el teorema de Gauss.

  \begin{theorem}
    Sea \(D\) dominio regular y \(\vec{F} \in C^1\left(\overline{D}\right)\) un campo de vectores.
    Entonces
    \begin{align}
      \int_{\partial D} \vec{F} \cdot \nu d \sigma
      =
      \int_D \nabla \cdot \vec{F} dx
    \end{align}
  \end{theorem}
  \begin{proof}
    Para cada \(x \in \overline{D} = \partial D \cup D\) consideremos \(U_x\) el entorno de \(x\) que da el Lema \ref{lemma:localDivergenceTheorem}.
    Así \(\{U_x : x \in \overline{D}\}\) es un cubrimiento por abiertos (en la topología relativa) del compacto \(\overline{D}\);
    por tanto, se puede obtener un subcubrimiento finito \(\{U_1, \dots, U_r\}\) de \(\overline{D}\).

    Supongamos que podemos construir una familia finita de funciones \(\phi_1, \dots, \phi_r\) de clase \(C^{\infty}_0(\overline{D})\), tales que
    \begin{enumerate}
      \item \(\phi_i \geq 0\);
      \item \(\support \phi_i \subseteq U_i\) (en particular \(\phi_i = 0\) fuera de \(U_i\));
      \item \(\sum_{i = 1}^r \phi_i(x) = 1\) si \(x \in \overline{D}\)
    \end{enumerate}
    Las familias \(\{\phi_1, \dots, \phi_r\}\) de funciones regulares que verifican estas tres propiedades reciben el nombre de \emph{particiones de la unidad subordinadas al cubrimiento} \(\{U_1, \dots, U_r\}\).

    Sea \(\vec{F} \in C^1(\overline{D})\) un campo;
    si consideramos la partición de la unidad \(\{\phi_1, \dots, \phi_r\}\), tenemos
    \(
      \vec{F}
      =
      \sum_{i = 1}^r \phi_i \vec{F}
      =
      \sum_{i = 1}^r \vec{F}_i
    \).
    Ahora \(\vec{F}_i = \phi_i \vec{F}\) está soportado en \(U_{x_i}\) y por construcción se tiene:
    \(
      \int_D \nabla \cdot \vec{F}_i dx
      =
      \int_{\partial D} \vec{F} \cdot \nu d \sigma
    \).
    Por tanto
    \begin{align}
      \int_D \nabla \cdot \vec{F} dx
      =
      \sum_{i = 1}^r \int_D \nabla \cdot \vec{F}_i dx
      =
      \sum_{i = 1}^r \int_{\partial D} \vec{F}_i \cdot \nu d \sigma
      =
      \int_{\partial D} \vec{F} \cdot \nu d \sigma
    \end{align}
    Módulo la construcción de particiones de la unidad hemos probado el teorema
  \end{proof}
\end{document}