\documentclass{article}
% Text Packages
\usepackage[spanish]{babel}
\usepackage[inline]{enumitem}

% Math Packages
\usepackage{mathtools}
\usepackage{amssymb, amsthm}
\usepackage{physics}
\usepackage{bbm}

\newcommand{\placeholderParameter}{-}
\newcommand{\characteristic}{\mathbbm{1}}
\newcommand{\naturalNumbers}{\mathbb{N}}
\newcommand{\realNumbers}{\mathbb{R}}
\newcommand{\integrableFunctions}{L^1(\realNumbers^n)}
\newtheorem{definition}{Definición}
\newtheorem{exercise}{Ejercicio}

\DeclareMathOperator{\lebesgueMeasure}{\lambda}

\theoremstyle{remark}
\newtheorem{remark}{Observación}

\title{Práctica 0 \\ Preliminares de Análisis}
\author{Pablo Brianese}
\begin{document}
\maketitle
  \begin{definition}
    Dadas \(f, g : \realNumbers^n \rightarrow \realNumbers\) ambas en \(L(\realNumbers^n)\), definimos la convolución de la siguiente manera
    \begin{align}
      (f * g) (x)
      =
      \int_{\realNumbers^n} f(y) g(x - y) \dd y
    \end{align}
  \end{definition}

  \begin{exercise}
    Probar que si \(f, g, h \in L^1(\realNumbers^n)\) y \(\lambda \in \realNumbers\), entonces valen:
    \begin{enumerate}
      \item 
        \label{exercise:convolutionConmutativity}
        \(f * g = g * f\)
      \item
        \label{exercise:convolutionDistributiveLaw}
        \(f * (g + h) = f * g + f * h\)
      \item 
        \label{exercise:convolutionAsociativity}
        \(f * (g * h) = (f * g) * h\)
      \item 
        \label{exercise:convolutionScalarAsociativity}
        \(\lambda (f * g) = f * (\lambda g)\)
      \item
        \label{exercise:convolutionL1Bound}
        \(\norm{f * g}_1 \leq \norm{f}_1 \norm{g}_1\)
    \end{enumerate}
  \end{exercise}

  \begin{remark}
    Las propiedades anteriores se pueden resumir diciendo que \((L^1(\realNumbers^n), \norm{\placeholderParameter}_1)\) es un álgebra de Banach conmutativa con la convolución como producto.
  \end{remark}

  \begin{proof}
    \ref{exercise:convolutionDistributiveLaw}
    Por la ley distributiva de los números reales y la linealidad de la integral, para todo \(x \in \realNumbers^n\)
    \begin{align}
      f * (g + h) (x)
      &=
      \int_{\realNumbers^n} f(y) (g + h) (x - y) \dd y
      \\
      &=
      \int_{\realNumbers^n} f(y) g(x - y) + f(y) h(x - y)  \dd y
      \\
      &=
      \int_{\realNumbers^n} f(y) g(x - y) \dd y + \int_{\realNumbers^n} f(y) h(x - y)  \dd y
      \\
      &=
      f * g (x) + f * h (x)
    \end{align}
  \end{proof} 

  \begin{proof}
    \ref{exercise:convolutionScalarAsociativity}
    Sea \(x \in \realNumbers^n\) arbitrario.
    Por la linealidad de la integral
    \begin{align}
      (\lambda (f * g)) (x)
      =
      \lambda \int_{\realNumbers^n} f(y) g(x - y) \dd y
      =
      \int_{\realNumbers^n} \lambda (f(y) g(x - y)) \dd y
    \end{align}
    Pero por asociatividad del producto entre números reales
    \begin{align}
      (\lambda f) * g (x)
      &=
      \int_{\realNumbers^n} (\lambda f(x)) g(x - y) \dd y
      =
      \int_{\realNumbers^n} \lambda (f(y) g(x - y)) \dd y
      \\
      f * (\lambda g) (x)
      &=
      \int_{\realNumbers^n} f(x) (\lambda g(x - y)) \dd y
      =
      \int_{\realNumbers^n} \lambda (f(y) g(x - y)) \dd y      
    \end{align}
    Comparando los extremos derechos de las desigualdades se obtiene \(\lambda (f * g) = (\lambda f) * g = f * (\lambda g)\).
  \end{proof}

  \begin{proof}
    \ref{exercise:convolutionL1Bound}
    Haremos un argumento por clases crecientes doble.
    Principalmente, consideraremos la clase \(F\) formada por las funciones \(f \in \integrableFunctions\) tales que \(\norm{f * g}_1 \leq \norm{f}_1 \norm{g}_1\) para toda \(g \in \integrableFunctions\).
    De forma accesoria, para cada \(f \in \integrableFunctions\), consideraremos el conjunto \(G_f\) formado por las funciones \(g \in \integrableFunctions\) tales que \(\norm{f * g}_1 \leq \norm{f}_1 \norm{g}_1\).

    Propiedades de \(F\). Propiedades de \(G\).

    Sea \(f = \characteristic_A\) la función característica de un conjunto de medida finita.
  \end{proof}

  \begin{proof}
    \ref{exercise:convolutionConmutativity}
    Haremos un argumento por clases crecientes doble.
    Principalmente, consideraremos la clase \(F\) formada por las funciones \(f \in L^1(\realNumbers^n)\) tales que \(f * g = g * f\) para toda \(g \in L^1(\realNumbers^n)\).

    Si \(f \in L^1(\realNumbers^n)\), definimos el conjunto \(G_f\) formado por las funciones \(g \in L^1(\realNumbers^n)\) tales que \(f * g = g * f\). Este objeto satisface
    \begin{enumerate*}
      \item \(g_1 + g_2 \in G\) si \(g_1, g_2 \in G\);
      \item \(\alpha g \in G\) si \(g \in G\) y \(\alpha \in \realNumbers\);
      \item \(\lim_{m \rightarrow \infty} g_m \in G\) para toda sucesión \(\{g_m\}_m \subseteq G\) puntualmente convergente tal que existe \(h \in \integrableFunctions\) con \(\abs{g_m} \leq h\) \((\forall m \in \naturalNumbers)\).
    \end{enumerate*}

    Supongamos que \(f \in L^1(\realNumbers^n)\) es la función característica \(f = \characteristic_A\) de un conjunto medible \(A \subseteq \realNumbers^n\).
    Accesoriamente, consideramos la clase \(G\) formada por las funciones \(g \in L^1(\realNumbers^n)\) tales que \(\characteristic_A * g = g * \characteristic_A\).
    Supongamos que también \(g\) es la función característica de un conjunto medible \(B \subseteq \realNumbers^n\).
    Observemos que \(\characteristic_C (x - y) = \characteristic_{x - C} (y)\) para todo par \(x, y \in \realNumbers^n\) y todo subconjunto \(C \subseteq \realNumbers^n\).
    A partir de esta relación pordemos calcular como, para todo \(x \in \realNumbers^n\)
    \begin{align}
      f * g (x)
      &=
      \int_{\realNumbers^n} \characteristic_A(y) \characteristic_B(x - y) \dd y
      =
      \int_{\realNumbers^n} \characteristic_A(y) \characteristic_{x - B} (y) \dd y
      \\
      &=
      \int_{\realNumbers^n} \characteristic_{A \cap (x - B)}(y) \dd y
      =
      \lebesgueMeasure(A \cap (x - B))
      \\
      g * f(x)
      &=
      \int_{\realNumbers^n} \characteristic_A(x - y) \characteristic_B(y) \dd y
      =
      \int_{\realNumbers^n} \characteristic_A(x - y) \characteristic_B(x - (x - y)) \dd y
      \\
      &=
      \int_{\realNumbers^n} \characteristic_A(x - y) \characteristic_{x - B} (x - y) \dd y
      =
      \int_{\realNumbers^n} \characteristic_{A \cap (x - B)}(x - y) \dd y
      \\
      &=
      \int_{\realNumbers^n} \characteristic_{x - A \cap (x - B)}(y) \dd y
      =
      \lebesgueMeasure(x - A \cap (x - B))
    \end{align}
    Ahora la ecuación que necesitamos se sigue de la invarianza por traslaciones (\(+ x\)) y reflexiones (\(-\)) de la medida de Lebesgue, algo que es facil de ver en el caso de los rectángulos que generan su \(\sigma\)--álgebra.
    Concretamente, sucede
    \(
      \lebesgueMeasure(A \cap (x - B))
      =
      \lebesgueMeasure(x - A \cap (x - B))
    \)
    para todo \(x \in \realNumbers^n\).
    Esta nos dice \(f * g = g * f\).
    Dado que \(g\) era arbitraria, toda función característica \(g \in L^1(\realNumbers^n)\) resulta ser un elemento de \(G\).

    Toda función característica de un conjunto de medida finita pertenece a la clase \(F\).

    Ahora bien, si \(\alpha, \beta \in \realNumbers\) son escalares y \(g_1, g_2 : \realNumbers\)
     
  \end{proof}
\end{document}