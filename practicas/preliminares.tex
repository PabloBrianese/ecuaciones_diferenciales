\documentclass{article}
% Text Packages
\usepackage[spanish]{babel}

% Math Packages
\usepackage{mathtools}
\usepackage{amssymb, amsthm}
\usepackage{physics}
\usepackage{bbm}

\newcommand{\placeholderParameter}{-}
\newcommand{\characteristic}{\mathbbm{1}}
\newcommand{\realNumbers}{\mathbb{R}}
\newtheorem{definition}{Definición}
\newtheorem{exercise}{Ejercicio}

\DeclareMathOperator{\lebesgueMeasure}{\lambda}

\theoremstyle{remark}
\newtheorem{remark}{Observación}

\title{Práctica 0 \\ Preliminares de Análisis}
\author{Pablo Brianese}
\begin{document}
\maketitle
  \begin{definition}
    Dadas \(f, g : \realNumbers^n \rightarrow \realNumbers\) ambas en \(L(\realNumbers^n)\), definimos la convolución de la siguiente manera
    \begin{align}
      (f * g) (x)
      =
      \int_{\realNumbers^n} f(y) g(x - y) \dd y
    \end{align}
  \end{definition}

  \begin{exercise}
    Probar que si \(f, g, h \in L^1(\realNumbers^n)\) y \(\lambda \in \realNumbers\), entonces valen:
    \begin{enumerate}
      \item 
        \label{exercise:convolutionConmutativity}
        \(f * g = g * f\)
      \item \(f * (g + h) = f * g + f * h\)
      \item \(f * (g * h) = (f * g) * h\)
      \item \(\lambda (f * g) = f * (\lambda g)\)
      \item \(\norm{f * g}_1 \leq \norm{f}_1 \norm{g}_1\)
    \end{enumerate}
  \end{exercise}

  \begin{remark}
    Las propiedades anteriores se pueden resumir diciendo que \((L^1(\realNumbers^n), \norm{\placeholderParameter}_1)\) es un álgebra de Banach conmutativa con la convolución como producto.
  \end{remark}

  \begin{proof}[Solución]
    \ref{exercise:convolutionConmutativity}
    Haremos un argumento por clases crecientes.

    Supongamos que \(f\), \(g\) son las funciones características de subconjuntos medibles \(A, B \subseteq \realNumbers^n\).
    Observemos que \(\characteristic_C (x - y) = \characteristic_{x - C} (y)\) para todo par \(x, y \in \realNumbers^n\) y todo subconjunto \(C \subseteq \realNumbers^n\).
    Luego para todo \(x \in \realNumbers^n\)
    \begin{align}
      f * g (x)
      &=
      \int_{\realNumbers^n} \characteristic_A(y) \characteristic_B(x - y) \dd y
      =
      \int_{\realNumbers^n} \characteristic_A(y) \characteristic_{x - B} (y) \dd y
      \\
      &=
      \int_{\realNumbers^n} \characteristic_{A \cap (x - B)}(y) \dd y
      =
      \lebesgueMeasure(A \cap (x - B))
      \\
      g * f(x)
      &=
      \int_{\realNumbers^n} \characteristic_A(x - y) \characteristic_B(y) \dd y
      =
      \int_{\realNumbers^n} \characteristic_A(x - y) \characteristic_B(x - (x - y)) \dd y
      \\
      &=
      \int_{\realNumbers^n} \characteristic_A(x - y) \characteristic_{x - B} (x - y) \dd y
      =
      \int_{\realNumbers^n} \characteristic_{A \cap (x - B)}(x - y) \dd y
      \\
      &=
      \int_{\realNumbers^n} \characteristic_{x - A \cap (x - B)}(y) \dd y
      =
      \lebesgueMeasure(x - A \cap (x - B))
    \end{align}
    Ahora la ecuación que necesitamos se sigue de la invarianza por traslaciones (\(+ x\)) y reflexiones (\(-\)) de la medida de Lebesgue, algo que es facil de ver en el caso de los rectángulos que generan su \(\sigma\)--álgebra.
    Concretamente, para todo \(x \in \realNumbers^n\)
    \begin{align}
      \lebesgueMeasure(A \cap (x - B))
      =
      \lebesgueMeasure(x - A \cap (x - B))
    \end{align}
    En conclusión \(f * g = g * f\), al menos en el caso de las funciones características.
    
    Supongamos \(f \geq 0\).
    Y pensemos en el caso en que \(g\) es la función característica, \(g = \characteristic_A\), de un conjunto boreliano \(A \subseteq \realNumbers^n\).
    
    Esto nos permite calcular, para todo \(x \in \realNumbers^n\), fórmulas muy similares para ambas convoluciones
    \begin{align}
      f * \characteristic_A (x)
      &=
      \int_{\realNumbers^n} f(y) \characteristic_A (x - y) \dd y
      \\
      &=
      \int_{\realNumbers^n} f(y) \characteristic_{x - A} (y) \dd y
      \\
      \characteristic_A * f (x)
      &=
      \int_{\realNumbers^n} f(x - y) \characteristic_A (y) \dd y
      \\
      &=
      \int_{\realNumbers^n} f(x - y) \characteristic_A (x - (x - y)) \dd y
      \\
      &=
      \int_{\realNumbers^n} f(x - y) \characteristic_{x - A} (x - y) \dd y
    \end{align}
    Sea \(\{f_n\}_n\) una sucesión de funciones simples nonegativas que convergen puntualmente a \(f\).
    Por el teorema de convergencia dominada
    \begin{align}
      \lim_{n \rightarrow \infty} f_n * \characteristic_A (x)
      &=
      \lim_{n \rightarrow \infty} \int_{\realNumbers^n} f_n(y)\characteristic_{x - A}(y) \dd y
      \\
      &=
      \int_{\realNumbers^n} f(y) \characteristic_{x - A}(y) \dd y
      \\
      &=
      f * \characteristic_A (x)
      && (\forall x \in \realNumbers^n)
      \\
      \lim_{n \rightarrow \infty} \characteristic_A * f_n (x)
      &=
      \lim_{n \rightarrow \infty} \int_{\realNumbers^n} f_n(x - y)\characteristic_{x - A}(x - y) \dd y
      \\
      &=
      \int_{\realNumbers^n} f(x - y) \characteristic_{x - A}(x - y) \dd y
      \\
      &=
      \characteristic_A * f (x)
      && (\forall x \in \realNumbers^n)
    \end{align}    
  \end{proof}
\end{document}